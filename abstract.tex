\thispagestyle{plain}
\begin{center}
    \Large
    {\Huge{\color{myblue}   Rainfall Prediction using Machine Learning technique}}
    
    \vspace{0.4cm}
        $ ${\LARGE \bf \sl Abstract}
\end{center}
\rule{\textwidth} {0.5pt}
In this project, we will use machine learning techniques in order to predict the rainfall happening and after that we will forecast the future rainfall for the next 30 days using ARIMA model(Auto Regressive Integrated Moving Average). One of the most difficult tasks in weather forecasting is predicting rainfall. Therefore, it is very important if we can estimate rainfall accurately and on time in advance since only then can we take safety measures in advance for ongoing building projects, transportation tasks, airline operations, agricultural tasks, and floods. Some of the external factors that can affect rainfall has been taken into accounts like temperature, humidity, wind direction , wind speed, season for predicting rainfall using machine learning techniques. Because of the harsh weather, it is growing harder to predict the amount of rain to fall. If we take an examples of countries like India which depend mostly on agriculture so here accurate prediction of rainfall is very important.\newline

Using machine learning approaches, this study suggests a revolutionary real-time rainfall forecast method. For forecasting rainfall, our suggested study employs 4 distinct supervised machine learning algorithm strategies, including the random forest method, logistics regression, K Nearest Neighbors, and XG Boost and Pycaret which is an auto machine learning libraries. Before employing the classification method, all pre-processing procedures, such as data cleaning and normalisation, were completed on the dataset. The ARIMA (Auto Regressive Integrated Moving Average)model is then used to anticipate future rainfall.Both the geographic characteristic as well as the time horizon comprised has major influence on the weather forecasting.LSTM(Long Short Term Memory) techniques in also used for forecasting rainfall but in our case it does not perform good in terms of giving the output.\newline
In terms of experimental result we see that XG Boast technique perform the best in term of accuracy with 86 percent the predicting the rainfall over other models like Logistics regression(83.86 percent), Random Forest Method(85.25 percent), Gaussian Naive Bayes Classifier (80.73 percent), K Nearesrt Neighbour(82.83 percent) and Latent Dirichlet Allocation(83.95 percent) .